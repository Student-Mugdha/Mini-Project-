\documentclass[16pt]{article}
\usepackage{graphicx}
\usepackage{geometry}
\usepackage{lipsum}

% Page setup
\geometry{a4paper, margin=1in}

\begin{document}
\begin{titlepage}
\centering

% University logo
\includegraphics[width=4cm]{logo.png} % Use your university logo file

% University name
\section{\vspace{1cm}}
{\Huge \textbf{Shri Govindram Seksariya Institute of Technology and Science, Indore}}
\vspace{1cm}

% Session
{\LARGE \textbf{Session : 2023-24}}
\vspace{1cm}

% Title
{\Huge \textbf{PONG GAME}}
\vspace{1cm}

% Mini Project
{\Huge \textbf{MINI PROJECT}}
\vspace{1cm}

% Submitted by
\begin{flushleft}
  \LARGE \textbf{SUBMITTED BY : MUGDHA DEOSKAR}
\end{flushleft}
\vspace{1cm}

% Enrollment number and submission date
\begin{flushleft}
  \LARGE \textbf{ENROLLMENT NUMBER : 0801CS221092}
  \vspace{2cm}
\end{flushleft}

\begin{flushleft}
  \LARGE \textbf{SUBMISSION DATE : 30/10/2323}
  \vspace{1cm}
\end{flushleft}
\end{titlepage}

% Start the introduction on the next page
\newpage

\section*{Introduction}

\begin{flushleft}
\Large The Pong Game is a classic two-player arcade game...
\end{flushleft}


\section*{Game Elements}
\section*{Game Window}
\begin{flushleft}
\Large The game is played in a graphical window with a dimension of 800 pixels in width and 400 pixels in height. It provides a visual representation of the playing field.
\end{flushleft}

\section*{Paddles}
\begin{flushleft}
\Large Two paddles, one for each player, are displayed vertically on the screen. Player 1 controls the left paddle, while Player 2 controls the right paddle. Players use designated keys to move their paddles up and down.
\end{flushleft}

\section*{Ball}
\begin{flushleft}
\Large A ball moves between the two paddles. The ball's position is tracked, and its speed can increase as the game progresses.
\end{flushleft}

\section*{Score}
\begin{flushleft}
\Large The game keeps track of each player's score. The objective is to score points by making the ball pass the opponent's paddle. The first player to reach 5 points wins the game.
\end{flushleft}

\section*{Timer}
\begin{flushleft}
\Large The game includes a timer that counts down from 3 minutes. If the timer runs out, the game ends. The timer's remaining time is displayed on the screen.
\end{flushleft}

\section*{User Interaction}
\begin{flushleft}
\Large Players use keyboard controls to interact with the game. Player 1 uses the "W" and "S" keys to move their paddle up and down, while Player 2 uses the "Up Arrow" and "Down Arrow" keys for the same purpose.
\end{flushleft}

\subsection*{Game Controls}
\begin{flushleft}
\Large Pressing the "SPACE" key starts the game.
\Large Pressing the "P" key pauses and unpauses the game.
\end{flushleft}

\section*{Sound (Currently Disabled)}
\begin{flushleft}
\Large The game has the functionality to play a bounce sound when the ball collides with the paddles. However, this feature is currently disabled in the code.
\end{flushleft}

\section*{Instructions}
\begin{flushleft}
\Large At the beginning of the game, instructions are displayed to inform the players about the controls and rules.
\end{flushleft}

\subsection*{Player Names}
\begin{flushleft}
\Large Players are prompted to enter their names at the start of the game, allowing for a personalized gaming experience.
\end{flushleft}

\subsection*{Winner Declaration}
\begin{flushleft}
\Large The game will declare a winner when one player scores 5 points or when the timer runs \Large out. The winner's name is displayed in a dialog box.
\end{flushleft}



\section*{How to Play}

\begin{flushleft}
\Large Start the game by pressing the "SPACE" key.
\Large Control your paddle using the designated keys:
\Large Player 1 (Left Paddle): "W" (Up) and "S" (Down)
\Large Player 2 (Right Paddle): "Up Arrow" (Up) and "Down Arrow" (Down)
\Large Try to bounce the ball past your opponent's paddle to score points.
\Large The game ends when one player reaches 5 points or when the timer runs out.
\end{flushleft}

\section*{Conclusion}

\begin{flushleft}
\Large The Pong Game is a simple yet enjoyable two-player arcade game that recreates the \Large excitement of a table tennis match. It offers interactive gameplay, customizable player \Large names, and a timer for added challenge. Although the sound feature is currently \Large disabled, the game provides a fun and engaging experience for players.
\end{flushleft}

\newpage

\section*{Meanings for the key parts of Pong Game code}
\begin{flushleft}
  \large import javax.swing;
    
  \large import java.awt;
    
  \large import java.awt.event.*;
    
  \large import java.util.Random;
     
  \large import java.io.*;

  \large 1) Import necessary Java libraries for creating a graphical Pong Game. This includes Swing for the GUI, AWT for graphical elements, and event handling.
\end{flushleft}

\section*{Declared Instance variables}
\begin{flushleft}
\Large Declare instance variables to store the game state:
\newline 1. ballX and ballY store the ball's coordinates.
\newline 2. paddle1Y and paddle2Y store the vertical positions of the paddles.
\newline 3. ballSpeedX and ballSpeedY represent the ball's speed.
\newline 4. player1Score and player2Score store the scores for the two players.
\newline 5. ballDirectionRight tracks the ball's direction.
\newline 6. gamePaused and gameStarted control the game's pause and start states.
\newline 7. gameTimer and maxGameTime manage the game's timer. gameTimer counts elapsed time, and maxGameTime sets the maximum game duration in seconds (3 minutes).
\newline 8. showInstructions is a flag that controls whether instructions are displayed at the beginning of the game.
\newline userCount is a static variable that keeps track of the number of users who have played the game
\end{flushleft}

\section*{Other Specification and Functionalities}
\begin{flushleft}
\Large Some methods like paint, keypresses, resetball, DisplayWinner, Lets start have been used to enhance made use of object-oriented programming (OOP) concepts.
\newline The main method is the entry point for the program. It sets up the game window and starts the game by creating a PongGame object and adding it to the JFrame.
\end{flushleft}

\newpage

\section*{In the provided code for the Pong Game, there are two libraries or frameworks mentioned: AWT (Abstract Window Toolkit) and Spring. Let's briefly overview what these are and their potential usage in the context of the game.}

\begin{itemize}
    \item \section*{1. AWT (Abstract Window Toolkit):}
    \begin{flushleft}
     \Large   - Overview: AWT is a part of the Java Foundation Classes (JFC) and provides a  \Large  platform-independent framework for building graphical user interfaces (GUIs) in  \Large  Java. It is one of the oldest GUI libraries in Java and is part of the Java  \Large  Standard Library.
     \Large      - Usage in Pong Game: AWT is used for creating the graphical user interface  \Large  of the Pong Game. In the code, you can see the usage of AWT components such as  \Large  `JFrame`, `JPanel`, and `Graphics` for creating and rendering the game window,  \Large  paddles, ball, and other graphical elements.
    \end{flushleft}
    \item \section*{2. Spring Framework (Possibly Spring Boot):}
    \begin{flushleft}
      \Large     - Overview: Spring is a widely-used framework for building Java-based  \Large  enterprise applications. It provides a comprehensive programming and configuration  \Large  model for modern Java-based enterprise applications, including aspects like  \Large  dependency injection, transaction management, and more. Spring Boot is a project  \Large  within the Spring ecosystem that simplifies the setup and development of Spring  \Large  applications.
       \Large    - Usage in Pong Game: While it's not explicitly mentioned in the code, Spring  \Large  or Spring Boot is not a typical choice for developing games like Pong. Spring  \Large  is more commonly used for building web applications, microservices, and  \Large   \Large  enterprise-level software. It's possible that the game uses other features of  \Large  Java, but Spring does not appear to be a primary component in the provided code.
    \end{flushleft}
\end{itemize}
\newpage

\section*{Debugging}

\begin{figure}[h]
  \centering
  \includegraphics[width=\textwidth]{debugging.png}
  \caption{Debugging}
\end{figure}

\newpage

\section*{Screenshots of Profiling}

\begin{figure}[h]
  \centering
  \includegraphics[width=\textwidth]{p1.png}
  \caption{Screenshot 1}
\end{figure}

\begin{figure}[h]
  \centering
  \includegraphics[width=\textwidth]{p2.png}
  \caption{Screenshot 2}
\end{figure}

\begin{figure}[h]
  \centering
  \includegraphics[width=\textwidth]{p3.png}
  \caption{Screenshot 3}
\end{figure}

\begin{figure}[h]
  \centering
  \includegraphics[width=\textwidth]{p4.png}
  \caption{Screenshot 4}
\end{figure}

\end{document}